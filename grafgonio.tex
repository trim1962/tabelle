\chapter{Goniometria}
\section{Angoli}
\begin{figure} %[htbp]
	\centering
	\includestandalone[width=.6\textwidth]{funzgonioTikz/angoliconcaviconvessi}
	\caption{Angoli concavi e convessi, positivi e negativi}
	\label{fig:angconconvposoneg}
	\end{figure}
	\begin{figure} %[htbp]
		\centering
		\includestandalone[width=\textwidth]{funzgonioTikz/angolinotevoli}
		\caption{Angoli notevoli}
		\label{fig:Angolorettoposneggonio}
		\end{figure}
		\begin{figure}
			\includestandalone[width=\textwidth]{funzgonioTikz/mappagomiometricaangolo}
			\caption{Mappa goniometria l'angolo}
			\label{fig:MappaGonometria1}
		\end{figure}
\section{Radianti}
\begin{figure}
	\centering
	\includestandalone[width=7.5cm]{funzgonioTikz/radianti}
	\caption{Radianti}
	\label{fig:radinatidefgonio}
\end{figure}
\section{Circonferenza goniometrica}
\begin{figure}
	\centering
	\includestandalone[width=7.5cm]{funzgonioTikz/circonferenzagoniometrica}
	\captionof{figure}{Circonferenza goniometrica}
	\label{fig:circonferenzagonimetricagonio}
\end{figure}
\begin{figure}
	\centering
	\includestandalone[width=20cm]{funzgonioTikz/CirGonioSpecial}
	\captionof{figure}{Circonferenza goniometrica}
	\label{fig:circonferenzagonimetricagonio2}
\end{figure}