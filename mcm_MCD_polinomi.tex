\chapter{MCD e mcm}
\label{cha:mcmmcdpolinomi}
\minitoc
\mtcskip                                % put some skip here
\minilof                                % a minilof
\mtcskip                                % put some skip here
\minilot
\section{MCD}\index{Polinomi!MCD}
\label{secMCDpolinomi}

\section{mcm}\index{Polinomi!mcm}
\label{sec:mcmpolinomi}
Il $\mcm$ fra più polinomi si ottiene moltiplicando i fattori comuni  e non comuni con il massimo esponente.

In genere, i polinomi sono una somma di più termini (addendi) quindi non è possibile,in genere, calcolare subito il $\mcm$.  
\begin{esempio}
Supponiamo di voler trovare il $\mcm$ fra $a^3-2a^2b$ e $a^2-4b^2$ i due polinomi sono somme di addendi quindi 
\begin{center}
\begin{tikzpicture}
\tikzset{notondo/.style={ellipse}};
\tikzset{tondo/.style={draw,notondo}};
\tikzset{linea/.style={very thick}};
\tikzset{freccia/.style={-triangle 90,linea}};
\tikzstyle{etichetta}=[tondo,node distance = 2
cm];
\tikzstyle{noetichetta}=[];
\tikzset{freccialta/.style={freccia,bend left=45}};
\tikzset{frecciabassa/.style={freccia,bend right=45}};
\node [notondo] (p1)  {$a^3$};
\node [notondo,right of=p1] (p2)  {$-$};
\node [notondo,right of=p2] (p3)  {$2a^2b$};
\node [notondo,right of=p3] (p4)  {};
\node [notondo,right of=p4] (p5)  {$a^2$};
\node [notondo,right of=p5 ] (p6)  {$-$};
\node [notondo,right of=p6] (p7)  {$4b^2$};
\node [etichetta,below of=p6] (p9)  {Addendi};
\node [etichetta,below of=p2] (p8)  {Addendi};
\draw[freccia] (p8)--(p1);
\draw[freccia] (p8)--(p3);
\draw[freccia] (p9)--(p5);
\draw[freccia] (p9)--(p7);
\end{tikzpicture}
\end{center}
raccogliendo a fattore comune nel primo e osservando che nel secondo abbiamo una differenza di quadrati otteniamo 
\begin{center}
\begin{tikzpicture}
\tikzset{notondo/.style={ellipse}};
\tikzset{tondo/.style={draw,notondo}};
\tikzset{linea/.style={very thick}};
\tikzset{freccia/.style={-triangle 90,linea}};
\tikzstyle{etichetta}=[tondo,node distance = 2cm];
\tikzstyle{noetichetta}=[];
\tikzset{freccialta/.style={freccia,bend left=45}};
\tikzset{frecciabassa/.style={freccia,bend right=45}};
\node [notondo,node distance=0.5cm] (p1)  {$a^2$};
\node [notondo,right of=p1] (p2)  {$$};
\node [notondo,right of=p2,,node distance=0.1cm] (p3)  {$(a-2b)$};
\node [notondo,right of=p3,node distance=2cm ] (p4)  {};
\node [notondo,right of=p4,] (p5)  {$(a-2b)$};
\node [notondo,right of=p5,,node distance=0.5cm ] (p6)  {$$};
\node [notondo,right of=p6] (p7)  {$(a+2b)$};
\node [etichetta,below of=p4] (p9)  {Fattori comuni};
\node [etichetta,above of=p4] (p8)  {Fattori non comuni};
\draw[freccia] (p8)--(p1);
\draw[freccia] (p8)--(p7);
\draw[freccia] (p9)--(p3);
\draw[freccia] (p9)--(p5);
\end{tikzpicture}
\end{center}
Abbiamo trasformato le due somme in un prodotto di fattori. Possiamo dividere i fattori  in fattori comuni e in fattori non comuni. Il $\mcm$ fra i due polinomi sarà\[a^2(a-2b)(a+2b) \]
\end{esempio}

