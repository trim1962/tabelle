\chapter{Trigonometria}
\label{cha:trigonometria}
\minitoc
\mtcskip                                % put some skip here
\minilof                                % a minilof
\mtcskip                                % put some skip here
\minilot
\begin{figure}
	\centering
	\includestandalone{trigonometria/triangolopitagorico1}
	\caption{Triangolo rettangolo}
	\label{fig:triangolopitagorico1}
\end{figure}
\section{I triangoli rettangoli}
Iniziamo con un po di notazione. I punti si indicano con le lettere maiuscole, la lunghezza dei segmenti con le lettere minuscole e le ampiezze degli angoli con le lettere greche. All'angolo di ampiezza corrisponde la lettera A. Le rimanenti partendo da A e muovendosi in senso antiorario si assegnano gli altri vertici. Un triangolo rettangolo è formato da due lati chiamati cateti\index{Triangolo!rettangolo!cateto} e un lato chiamato ipotenusa.

 L'ipotenusa\index{Triangolo!rettangolo!ipotenusa} è il lato di lunghezza maggiore. I lati di un triangolo sono classificati, rispetto ad un angolo, come opposti o adiacenti. Guardando la figura\nobs\vref{fig:triangolooppostoadiacente}, rispetto all'angolo $\gamma$ il cateto AC è adiacente perché lato dell'angolo, mentre il cateto AB è opposto perché non fa parte dell'angolo. 
 
 Per convenzione, si dice che il cateto opposto all'angolo $\alpha$ ha lunghezza a, il lato opposto all'angolo $\beta$ ha lunghezza b e infine, il lato opposto al vertice C ha lunghezza C. La figura\nobs\vref{fig:triangolopitagorico1} mostra come devono essere assegnati i nomi.
 
 La somma degli angoli interni di un triangolo è un angolo piatto. Quindi $\alpha+\beta+\gamma=\ang{180}$. 
\begin{figure}
	\centering
	\includestandalone{trigonometria/triangolooppostoadiacente}
	\caption{Elementi di un triangolo rettangolo}
	\label{fig:triangolooppostoadiacente}
\end{figure}
\subsection{Relazioni fondamentali}
Valgono le seguenti relazioni fra i cateti gli angoli e l'ipotenusa
\begin{align*}
c&=a\sen\gamma&b&=a\cos\gamma\\
b&=a\sen\beta&c&=a\cos\beta
\end{align*}
Quindi

Un cateto è uguale  al prodotto dell'ipotenusa per il seno dell'angolo opposto

oppure

Un cateto è uguale al prodotto dell'ipotenusa per il coseno dell'angolo opposto.

Valgono le seguenti relazioni
\begin{align*}
\dfrac{c}{a}&=\sen\gamma&\dfrac{b}{a}&=\cos\gamma\\
\dfrac{b}{a}&=\sen\beta&\dfrac{c}{a}&=\cos\beta
\end{align*}
\altapriorita{Esempi pratici}
\altapriorita{Teoremi dei seni}
\altapriorita{Teorema di Carnot}
