\chapter{Trigonometria}
\label{cha:trigonometria}
\minitoc
\mtcskip                                % put some skip here
\minilof                                % a minilof
\mtcskip                                % put some skip here
\minilot
\begin{figure}
	\centering
	\includestandalone{trigonometria/triangolopitagorico1}
	\caption{Triangolo rettangolo}
	\label{fig:triangolopitagorico1}
\end{figure}
\section{I triangoli rettangoli}
Iniziamo con un po di notazione. I punti si indicano con le lettere maiuscole, la lunghezza dei segmenti con le lettere minuscole e le ampiezze degli angoli con le lettere greche. All'angolo di ampiezza corrisponde la lettera $A$. Per rimanenti partendo da $A$ e muovendosi in senso antiorario, si assegnano gli altri vertici. Un triangolo rettangolo è formato da due lati chiamati cateti\index{Triangolo!rettangolo!cateto} e un lato chiamato ipotenusa. L'ipotenusa\index{Triangolo!rettangolo!ipotenusa} è il lato di lunghezza maggiore. 

I lati di un triangolo sono classificati, rispetto ad un angolo, come opposti o adiacenti. Guardando la figura\nobs\vref{fig:triangolooppostoadiacente}, rispetto all'angolo $\gamma$ il cateto\index{Cateto!opposto}\index{Cateto!adiacente} AC è adiacente perché lato dell'angolo, mentre il cateto AB è opposto\index{Angolo!opposto}\index{Angolo!adiacente}  non facendo parte dell'angolo. 
 
 Per convenzione, si dice che il lato opposto all'angolo $\alpha$ ha lunghezza $a$, il lato opposto a $\beta$ è lungo $b$ e infine, il lato opposto al vertice C ha lunghezza $c$. La figura\nobs\vref{fig:triangolopitagorico1} mostra come devono essere assegnati i nomi.
 
 La somma degli angoli interni\index{Angoli!interni!somma} di un triangolo è un angolo piatto. Quindi $\alpha+\beta+\gamma=\ang{180}$. 
\begin{figure}
	\centering
	\includestandalone{trigonometria/triangolooppostoadiacente}
	\caption{Elementi di un triangolo rettangolo}
	\label{fig:triangolooppostoadiacente}
\end{figure}
\subsection{Relazioni fondamentali}
Consideriamo la figura\nobs\vref{fig:triangolopitagorico1}. Valgono le seguenti relazioni fra i cateti gli angoli e l'ipotenusa
\begin{align*}
c&=a\sen\gamma&b&=a\cos\gamma\\
b&=a\sen\beta&c&=a\cos\beta
\end{align*}
Quindi

Un cateto è uguale  al prodotto dell'ipotenusa per il seno dell'angolo opposto

\noindent oppure

Un cateto è uguale al prodotto dell'ipotenusa per il coseno dell'angolo opposto.

\noindent Valgono le seguenti relazioni
\begin{align*}
\dfrac{c}{a}&=\sen\gamma&\dfrac{b}{a}&=\cos\gamma\\
\dfrac{b}{a}&=\sen\beta&\dfrac{c}{a}&=\cos\beta
\end{align*}
Quindi

Il rapporto fra  cateto e l'ipotenusa è uguale al seno dell'angolo opposto

\noindent oppure

Il rapporto fra  cateto e l'ipotenusa è uguale al coseno dell'angolo adiacente.

\noindent Dalle relazioni di partenza si ottiene
\[\dfrac{b}{sen\beta}=\dfrac{b}{\cos\gamma}=\dfrac{c}{\sen\gamma}=\dfrac{c}{cos\beta}=a \]

Dividendo fra loro le relazioni di partenza otteniamo
\begin{align*}
b&=c\tg\beta&c&=b\tg\gamma\\
b&=c\cotg\gamma&c&=b\cotg\beta
\end{align*}
\section{Risoluzione triangoli rettangoli}
La risoluzione di un triangolo consiste nel trovare tutti gli elementi di un triangolo conoscendone alcuni. Prima di iniziare è importante ricordare che in un triangolo rettangolo vale il teorema di Pitagora\index{Teorema!Pitagora} quindi, facendo riferimento alla figura\nobs\vref{fig:TeoremaPitagora_1}:
\begin{align*}
a^2&=b^2+c^2\\
a{}&=\sqrt{b^2+c^2}\\
b^2&=a^2-c^2\\
b{}&=\sqrt{a^2-c^2}\\
c^2&=a^2-b^2\\
c{}&=\sqrt{a^2-b^2}\\
\end{align*}
\begin{figure}
	\centering
	\includestandalone{trigonometria/pitagora_1}
	\caption{Teorema di Pitagora}
	\label{fig:TeoremaPitagora_1}
\end{figure}
Inoltre dato che la somma degli angoli interni di un triangolo è 
$\ang{180}$ la somma dei due angoli acuti è di $\ang{90}$. quindi
\[\beta+\gamma=\ang{90}\]
\subsection{Angolo acuto e ipotenusa noti}
Risolviamo questo caso, conosciamo, come nella figura\nobs\vref*{fig:risTriangRett_1}, l'ipotenusa $a$ e un angolo acuto, per esempio $\gamma$:
\begin{align*}
\beta&=\ang{90}-\gamma&\beta&=\ang{90}-\gamma\\
c&=a\sen\gamma&c&=a\cos\beta\\
b&=a\cos\gamma&b&=a\cos\beta
\end{align*}
\subsection{Angolo acuto e cateto noti}
Risolviamo questo caso, conosciamo, come nella figura\nobs\vref*{fig:risTriangRett_2}, un cateto $b$ e un angolo acuto, per esempio $\gamma$:
\begin{align*}
\beta&=\ang{90}-\gamma&\beta&=\ang{90}-\gamma\\
c&=b\tg\gamma&c&=a\cotg\beta\\
a&=\dfrac{b}{\sen\beta}&a&=\dfrac{b}{\cos\beta}
\end{align*}
\subsection{Ipotenusa e cateto}
Risolviamo questo caso, conosciamo, come nella figura\nobs\vref*{fig:risTriangRett_3}, un cateto $b$ e l'ipotenusa $a$:
\begin{align*}
\beta&=\arcsen\dfrac{b}{a}&\gamma&=\arccos\dfrac{b}{a}\\
\gamma&=\ang{90}-\beta&\beta&=\ang{90}-\gamma\\
c&=b\cos\beta&c&a\sen\gamma
\end{align*}
\subsection{Cateti noti}
Risolviamo questo caso, conosciamo, come nella figura\nobs\vref*{fig:risTriangRett_4}, con i cateti $b$ e $c$ noti:
\begin{align*}
\gamma&=\arctg\dfrac{c}{b}&\beta&=\arctg\dfrac{b}{c}\\
\beta&=\ang{90}-\gamma&\gamma&=\ang{90}-\beta\\
a&=\dfrac{c}{\sen\gamma}&a&=\dfrac{c}{\cos\beta}
\end{align*}
\begin{figure}
	\begin{subfigure}[b]{.5\linewidth}
		\centering
\includestandalone{trigonometria/risTriangRett_1}
	\caption{Ipotenusa e angolo acuto noto}
	\label{fig:risTriangRett_1}
	\end{subfigure}%
	\begin{subfigure}[b]{.5\linewidth}
		\centering
		\includestandalone{trigonometria/risTriangRett_2}
		\caption{Cateto e angolo acuto noto}
		\label{fig:risTriangRett_2}
	\end{subfigure}
	\begin{subfigure}[b]{.5\linewidth}
		\centering
	\includestandalone{trigonometria/risTriangRett_3}
	\caption{Ipotenusa e cateto noto}
	\label{fig:risTriangRett_3}
	\end{subfigure}%
	\begin{subfigure}[b]{.5\linewidth}
		\centering
		\includestandalone{trigonometria/risTriangRett_4}
		\caption{Cateti noti}
		\label{fig:risTriangRett_4}
	\end{subfigure}
	\captionof{figure}{Risoluzione triangoli rettangoli}
	\label{fig:RisoluzioneTriangoliRettangoli}
\end{figure}
\section{Triangoli qualunque}
\subsection{Teorema dei seni}
\begin{figure}
	\centering
	\includestandalone{trigonometria/TeoSeni_1}
	\caption{Teorema dei seni}
	\label{fig:TeoremDeiSeni}
\end{figure}
Per un triangolo qualunque\index{Teorema!seni} valgono le seguenti uguaglianze\[\dfrac{a}{\sen\alpha}=\dfrac{b}{\sen\beta}=\dfrac{c}{\sen\beta}=2R \]

Quindi in un triangolo qualunque il rapporto fra il lato e il seno dell'angolo opposto è costante ed è uguale al diametro della circonferenza circoscritta. 

La figura\nobs\vref{fig:TeoremDeiSeni} mostra le relazioni.
\subsection{Teorema di Carnot}
\begin{figure}
	\centering
	\includestandalone{trigonometria/canot_1}
	\caption{Teorema di Carnot}
	\label{fig:TeoremDiCarnot_1}
\end{figure}
Una versione più generale del teorema di Pitagora è il teorema di Carnot\index{Teorema!Carnot}. Partendo dalla figura\nobs\vref{fig:TeoremDiCarnot_1} avremo queste relazioni:
\begin{align*}
a^2&=b^2+c^2-2bc\cos\alpha\\
b^2&=a^2+c^2-2ac\cos\beta\\
c^2&=a^2+b^2-2ab\cos\gamma
\end{align*}
Possiamo scrivere le precedenti equazioni isolando i coseni cioè:
\begin{align*}
\cos\alpha&=\dfrac{b^2+c^2-a^2}{2bc}\\
\cos\beta&=\dfrac{a^2+c^2-b^2}{2ac}\\
\cos\gamma&=\dfrac{a^2+b^2-c^2}{2ab}\\
\end{align*}
queste soluzioni possono servire per definire gli angoli del triangolo cioè:
\begin{align*}
\alpha&=\arccos(\dfrac{b^2+c^2-a^2}{2bc})\\
\beta&=\arccos(\dfrac{a^2+c^2-b^2}{2ac})\\
\gamma&=\arccos(\dfrac{a^2+b^2-c^2}{2ab})\\
\end{align*}
\begin{figure}
	\begin{subfigure}[b]{.5\linewidth}
		\centering
\includestandalone{trigonometria/risTriangQualunque_1}
	\caption{Un lato e due angoli noti}
	\label{fig:risTriangQqualunque_1}
	\end{subfigure}%
	\begin{subfigure}[b]{.5\linewidth}
		\centering
	\includestandalone{trigonometria/risTriangQualunque_2}
		\caption{Due lati l'angolo fra loro compreso noti}
		\label{fig:risTriangQqualunque_2}
	\end{subfigure}
	\begin{subfigure}[b]{.5\linewidth}
		\centering
		\includestandalone{trigonometria/risTriangQualunque_3}
		\caption{Due lati l'angolo fra loro non compreso noti}
		\label{fig:risTriangQqualunque_3}
	\end{subfigure}%
	\begin{subfigure}[b]{.5\linewidth}
		\centering
		\includestandalone{trigonometria/risTriangQualunque_4}
		\caption{Due lati l'angolo fra loro non compreso noti}
		\label{fig:risTriangQqualunque_4}
	\end{subfigure}
	\captionof{figure}{Risoluzione triangoli rettangoli}
	\label{fig:RisoluzioneTriangoliQualunque}
\end{figure}
\section{Risoluzione di triangolo qualunque}
\subsection{Un lato e due angoli}
Se è noto il lato $c$ e gli angoli $\alpha$ e $\beta$ come nella figura\nobs\vref{fig:risTriangQqualunque_1} avremo:
\begin{align*}
\gamma&=\ang{180}-(\alpha+\beta)\\
a&=c\dfrac{\sen\alpha}{\sen\gamma}\\
b&=c\dfrac{\sen\beta}{\sen\gamma}\\
\end{align*}
\subsection{Due lati e l'angolo fra essi compreso} 
In questo caso supponiamo noti i lati $b$ e $c$ e l'angolo $\alpha$ fra loro compreso, come nella figura\nobs\vref{fig:risTriangQqualunque_2} avremo:
\begin{align*}
a&=\sqrt{b^2+c^2-2bc\cos\alpha}\\
\beta&=\arccos(\dfrac{a^2+c^2-b^2}{2ac})\\
\gamma&=\arccos(\dfrac{a^2+b^2-c^2}{2ab})
\intertext{verificando che}
\alpha+&\beta+\gamma=\ang{180}
\end{align*}
\subsection{Due lati e l'angolo apposto a quello compreso}
In questo caso supponiamo noti i lati $b$ e $c$ e l'angolo $\beta$ fra loro non compreso, come nella figura\nobs\vref{fig:risTriangQqualunque_3} avremo:
\begin{align*}
\intertext{verificando le soluzioni}
\gamma&=\arcsen(\dfrac{c}{b}{\sen\beta}) \\
\alpha&=\ang{180}-(\beta+\gamma)\\
a&=b\dfrac{\sen\alpha}{\sen\beta}
\end{align*}
\subsection{Tre lati}
In questo caso supponiamo noti i lati $a$ $b$ e $c$  come nella figura\nobs\vref{fig:risTriangQqualunque_4} avremo:
\begin{align*}
\alpha&=\arccos(\dfrac{b^2+c^2-a^2}{2bc})\\
\beta&=\arccos(\dfrac{a^2+c^2-b^2}{2ac})\\
\gamma&=\arccos(\dfrac{a^2+b^2-c^2}{2ab})\\
\end{align*}