\chapter{Trigonometria}
\label{cha:trigonometria}
\minitoc
\mtcskip                                % put some skip here
\minilof                                % a minilof
\mtcskip                                % put some skip here
\minilot
\begin{figure}
	\begin{subfigure}[b]{.5\linewidth}
		\centering\begin{tikzpicture}[line cap=round,line join=round,>=triangle 45,x=1.0cm,y=1.0cm]
\clip(-0.5,-0.34) rectangle (5,3.5);
\draw [shift={(0,0)},fill=black,fill opacity=0.15] (0,0) -- (0:0.52) arc (0:36.87:0.52) -- cycle;
\draw (4,3)-- (0,0);
\draw (0,0)-- (4,0);
\draw (4,0)-- (4,3);
\draw (0.74,0.38) node[anchor=north west] {$\mathbf{\alpha}$};
\fill [color=black] (4,0) circle (1.5pt);
\draw[color=black] (4.06,-0.2) node {$A$};
\fill [color=black] (4,3) circle (1.5pt);
\draw[color=black] (4.06,3.3) node {$B$};
\fill [color=black] (0,0) circle (1.5pt);
\draw[color=black] (-0.22,-0.2) node {$C$};
\draw[color=black] (1.93,1.65) node {$a$};
\draw[color=black] (2.03,-0.2) node {$b$};
\draw[color=black] (4.15,1.55) node {$c$};
\end{tikzpicture}

		\caption{Triangolo rettangolo}\label{fig:TrigTriangoloRettangolo}
	\end{subfigure}%
	\begin{subfigure}[b]{.5\linewidth}
		\centering
		\begin{tabular}{L}
			\toprule
			b=a\cos\alpha\\
			c=a\sen\alpha\\
			c=b\tg\alpha\\
			b=c\cotg\alpha\\
			\bottomrule
		\end{tabular}
		\caption{Relazioni triangolo rettangolo}\label{fig:TeoremiTrinagoloRettangolo}
	\end{subfigure}
\caption{Trigonometria triangolo rettangolo}
\label{tab:trigonometriatriangolorettangolo}
\end{figure}
%\begin{table}[H]
%\centering
%\subfloat[][]{
%\begin{tikzpicture}[line cap=round,line join=round,>=triangle 45,x=1.0cm,y=1.0cm]
\clip(-0.5,-0.34) rectangle (5,3.5);
\draw [shift={(0,0)},fill=black,fill opacity=0.15] (0,0) -- (0:0.52) arc (0:36.87:0.52) -- cycle;
\draw (4,3)-- (0,0);
\draw (0,0)-- (4,0);
\draw (4,0)-- (4,3);
\draw (0.74,0.38) node[anchor=north west] {$\mathbf{\alpha}$};
\fill [color=black] (4,0) circle (1.5pt);
\draw[color=black] (4.06,-0.2) node {$A$};
\fill [color=black] (4,3) circle (1.5pt);
\draw[color=black] (4.06,3.3) node {$B$};
\fill [color=black] (0,0) circle (1.5pt);
\draw[color=black] (-0.22,-0.2) node {$C$};
\draw[color=black] (1.93,1.65) node {$a$};
\draw[color=black] (2.03,-0.2) node {$b$};
\draw[color=black] (4.15,1.55) node {$c$};
\end{tikzpicture}

%}
%\subfloat[][]{
%\begin{tabular}{L}
%\toprule
%b=a\cos\alpha\\
%c=a\sen\alpha\\
%c=b\tg\alpha\\
%b=c\cotg\alpha\\
%\bottomrule
%\end{tabular}
%}
%
%\caption{Trigonometria triangolo rettangolo}
%\label{tab:trigonometriatriangolorettangolo}
%\end{table}
\altapriorita{Esempi pratici}
\altapriorita{Teoremi dei seni}
\altapriorita{Teorema di Carnot}
