\chapter{Raccoglimento in fattori}
\label{cha:raccoglimentoinfattori}
\minitoc
\mtcskip                                % put some skip here
\minilof                                % a minilof
\mtcskip                                % put some skip here
\minilot
\begin{table}[H]
\centering
\begin{tabular}{lL}
\toprule
\multicolumn{2}{c}{Raccoglimenti}\\
\midrule
\multicolumn{1}{l}{Tipo}&\multicolumn{1}{l}{Esempio}\\
\midrule
totale&ab+ac=a(b+c)\\
\midrule
\multirow{3}*{parziale}&ab+ac+db+dc=\\
&a\underbrace{(b+c)}+d\underbrace{(b+c)}=\\
&=(b+c)(a+d)\\
\midrule
\multirow{3}*{quadrato binomio}&a^2+2ab+b^2=(a+b)^2\\
&\\
&a^2-2ab+b^2=(a-b)^2\\
\midrule
quadrato trinomio&a^2+b^2+c^2+2ab+2ac+2bc=(a+b+c)^2\\
\midrule
\multirow{3}*{cubo binomio}&a^3+b^3+3a^2b+3ab^2=(a+b)^3\\
&\\
&a^3-b^3-3a^2b+3ab^2=(a-b)^3\\
\midrule
differenza di quadrati&a^2-b^2=(a-b)(a+b)\\
\midrule
\multirow{3}*{Somma differenza cubi}&a^3-b^3=(a-b)(a^2+ab+b^2)\\
&\\
&a^3+b^3=(a+b)(a^2-ab+b^2)\\
\midrule
\multirow{3}*{trinomi 
particolari}&x^2+sx+p=(x+a)(x+b)\;\begin{cases}s=a+b\\ p=a\cdot b\end{cases}\\
&\\
&x^4+sx^2+p=(x^2+a)(x^2+b)\: \begin{cases}s=a+b\\ p=a\cdot b\end{cases}\\
\bottomrule
\end{tabular}
\caption{Polinomi raccoglimenti}
\label{tab:polinomiraccoglimenti}
\end{table}

