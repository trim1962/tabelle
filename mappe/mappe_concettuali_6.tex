\documentclass{standalone}
\input{../Mod_base/grafica}
\tikzset{main node/.style={rectangle, draw,thick,
 text width=7em, text centered, rounded corners, minimum height=4em}, }
\tikzset{main verb/.style={minimum size=1cm}, }
\tikzset{linea/.style={-triangle 90,thick}, }
\begin{document}
 \begin{tikzpicture}
 \node[main node] (1) {Potenza};
 \node[main verb] (2) [right=of 1] {\'{e}};
 \node[main node] (3) [right=of 2] {Operazione};
 \node[main verb] (4) [above=of 3] {ha};
 %\node[main verb] (5) [below right=of 3] {ha}; 
 \node[main verb] (6) [below =8cm and 0cm of 3] {ha};
 \node[main node] (7) [above= of 4] {Propriet\'{a}};
 
 \node[main node] (8) [right= of 6] {Casi particolari};
 
 \node[main node] (9) [left= of 6] {una base};
 
 \node[main node] (10) [below = of 6] {esponente};

 \node[main verb] (11) [left=0cm and 4cm of 7] {moltiplicare};
 \node[main verb] (12) [above= of 7] {potenza}; 
 \node[main verb] (13) [right=0cm and 4cm of 7] {dividere};
 
 \node[main verb] (14) [right= of 8] {sono};
 \node[main verb] (15) [left= of 9] {\'{e}};
 \node[main verb] (16) [left= of 10] {\'{e}};
 
 \node[main node] (17) [above left= of 11] {basi diverse ma uguale esponenete};
 \node[main node] (18) [below left= of 11] {Basi uguali};
 
 \node[main node] (19) [above= of 12] {Potenza di potenza};
 
 \node[main node] (20) [above right= of 13] {basi diverse ma uguale esponenete};
 \node[main node] (21) [below right= of 13] {Basi uguali};
 
 \node[main node] (22) [above right=of 14] {base zero e esponente zero};
 \node[main node] (23) [below right=of 14] {Base diversa da zero e esponente zero};
 \node[main node] (24) [left= of 15] {Un numero};
 \node[main node] (25) [left= of 16] {Un indice}; 
 
 \node[main verb] (26) [left=of 17] {\'{e}};
 \node[main verb] (27) [left=of 18] {\'{e}};
 \node[main verb] (28) [above=of 19] {\'{e}};
 \node[main verb] (29) [right=of 20] {\'{e}};
 \node[main verb] (30) [right=of 21] {\'{e}};
 \node[main verb] (31) [right =of 22] {non ha};
 \node[main verb] (32) [right=of 23] {ha};
 \node[main node] (33) [left=of 26] {Una potenza};
 \node[main node] (34) [left=of 27] {Una potenza};
 \node[main node] (35) [above=of 28] {Una potenza};
 \node[main node] (36) [right=of 29] {Una potenza};
 \node[main node] (37) [right=of 30] {Una potenza};
 \node[main node] (38) [right=of 31] {Significato};
 \node[main node] (39) [right=of 32] {Valore uno};
 \node[main verb] (40) [above=of 33] {ha};
 \node[main verb] (41) [below=of 34] {ha};
 \node[main verb] (42) [above=of 35] {ha};
 \node[main verb] (43) [above=of 36] {ha};
 \node[main verb] (44) [below=of 37] {ha};
 \node[main node] (45) [above left= of 40] {Base uguale al prodotto delle basi};
 \node[main node] (46) [above right= of 40] {Esponente uguale}; 
 \node[main node] (47) [below right= of 41] {Per esponente la somma degli esponenti };
 \node[main node] (48) [below left= of 41] {Base uguale};
 
 \node[main node] (49) [right=of 42] {Per esponente il prodotto degli esponenti };
 \node[main node] (50) [left= of 42] {Base uguale};
\node[main node] (51) [above right= of 43] {Esponente uguale };
 \node[main node] (52) [above left= of 43] {Base uguale al quoziente delle basi};
\node[main node] (53) [below left= of 44] {Esponente uguale };
 \node[main node] (54) [below right= of 44] {Base uguale al quoziente delle basi};

\foreach \x /\y in{1/2,2/3,3/4,3/6,4/7,6/8,6/9,6/10,7/11,7/12,7/13,8/14,9/15,10/16,11/17,11/18,12/19,13/20,13/21,14/22,14/23,15/24,16/25,17/26,18/27,19/28,20/29,21/30,22/31,23/32,26/33,27/34,28/35,29/36,30/37,31/38,32/39,33/40,34/41,35/42,36/43,37/44,40/45,40/46,41/47,41/48,42/49,42/50,43/51,43/52,44/53,44/54}
 \path[linea] (\x) edge node {} (\y);
% 
\end{tikzpicture}
\end{document}