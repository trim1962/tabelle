\documentclass[preview=true]{standalone}
\input{../Mod_base/grafica}
\input{../Mod_base/base}

\tikzset{main base/.style={draw,thick,text centered, }, }
\tikzset{main node/.style={rectangle, draw,rounded corners,main base}, }
%      
%\tikzset{main verb/.style={minimum size=1cm}, }
%\tikzset{linea/.style={-triangle 90}, }
%\tikzset{main node2/.style={rectangle, draw,thick,
%    text width=7em, text centered, rounded corners, minimum height=3em}, }
\tikzset{main node2/.style={main node,
    text width=7em, minimum height=3em}, }
\tikzset{main verb/.style={minimum size=1cm}, }
\tikzset{linea/.style={-triangle 90,thick,draw}}
\tikzset{primo/.style={circle,draw,inner sep=0pt,minimum size=1pt,thick}, }
\tikzset{
    start-end/.style={
        draw,
        rectangle,
        rounded corners,main base,
    } ,
    input/.style={ % requires library shapes.geometric
        draw,
        trapezium,
        trapezium left angle=60,
        trapezium right angle=120,main base
    },
    operation/.style={
        draw,thick,
        rectangle,main base,
    },
    loop/.style={ % requires library shapes.misc
        draw,
        chamfered rectangle,
        chamfered rectangle xsep=2cm
    },
    decision/.style={ % requires library shapes.geometric
        draw,
        diamond,
        aspect=#1,main base
    },
    decision/.default=1,
    print/.style={ % requires library shapes.symbols
        draw,
        tape,
        tape bend top=none
    },
    connection/.style={
        draw,
        circle,
        radius=5pt,
    },
    process rectangle outer width/.initial=0.15cm,
    predefined process/.style={
        rectangle,
        draw,
        append after command={
        \pgfextra{
          \draw
          ($(\tikzlastnode.north west)-(0,0.5\pgflinewidth)$)--
          ($(\tikzlastnode.north west)-(\pgfkeysvalueof{/tikz/process rectangle outer width},0.5\pgflinewidth)$)--
          ($(\tikzlastnode.south west)+(-\pgfkeysvalueof{/tikz/process rectangle outer width},+0.5\pgflinewidth)$)--
          ($(\tikzlastnode.south west)+(0,0.5\pgflinewidth)$);
          \draw
          ($(\tikzlastnode.north east)-(0,0.5\pgflinewidth)$)--
          ($(\tikzlastnode.north east)+(\pgfkeysvalueof{/tikz/process rectangle outer width},-0.5\pgflinewidth)$)--
          ($(\tikzlastnode.south east)+(\pgfkeysvalueof{/tikz/process rectangle outer width},0.5\pgflinewidth)$)--
          ($(\tikzlastnode.south east)+(0,0.5\pgflinewidth)$);
        }  
        },
        text width=#1,
        align=center
    },
    predefined process/.default=1.75cm,
    man op/.style={ % requires library shapes.geometric
        draw,
        trapezium,
        shape border rotate=180,
        text width=2cm,
        align=center,
    },
    extract/.style={
        draw,
        isosceles triangle,
        isosceles triangle apex angle=60,
        shape border rotate=90
    },
    merge/.style={
        draw,
        isosceles triangle,
        isosceles triangle apex angle=60,
        shape border rotate=-90
    },
}
\begin{document}
  \begin{tikzpicture}
    \node[main node2] (1) {Sottazione};
    \node[main verb] (2) [right =of 1]  {\'{e}};
    \node[main node2] (3) [right =of 2] {Operazione};
\node[main verb] (4) [above right= .3cm and 1cm  of 3]  {\'{e}};
\node[main verb] (5) [below right= .1cm and 1cm of 3]  {ha};
\node[main verb] (6) [above left  =of 3]  {non è\'{e}};
\node[main verb] (7) [below left =of 3]  {ha};
\node[main node2] (8) [above right = .01cm and 1cm of 6]  {Commutativa};
\node[main node2] (9) [above left = .01cm and 1cm of 6]  {Associativa};
\node[main node2] (10) [above right= .01cm and 1cm of 4]  {Invariantiva};
\node[main node2] (11) [above right = .0cm and 1cm of 5]  {Primo termine};
\node[main node2] (12) [below right = .0cm and 1cm of 5]  {Secondo termine};
\node[main node2] (13) [below left = of 7]  {Un risultato};
\node[main verb] (14) [right = of 11]  {\'{e}};
\node[main verb] (15) [right = of 12]  {\'{e}};
\node[main verb] (16) [right = of 13]  {\'{e}};
\node[main node2] (17) [right =of 14] {Il minuendo};
\node[main node2] (18) [right =of 15] {Il sottraendo};
\node[main node2] (19) [right =of 16] {La differenza};
\foreach \x /\y in{1/2,2/3,3/6,3/4,3/5,3/7,6/8,6/9,4/10,5/12,5/12,7/13,11/14,12/15,13/16,14/17,15/18,16/19,5/11}
  \path[linea] (\x) edge node {} (\y);
  
\end{tikzpicture}
\end{document}