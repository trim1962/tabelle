\documentclass[preview=true]{standalone}
\input{../Mod_base/grafica}
\input{../Mod_base/base}

\tikzset{main base/.style={draw,thick,text centered, }, }
\tikzset{main node/.style={rectangle, draw,rounded corners,main base}, }
%      
%\tikzset{main verb/.style={minimum size=1cm}, }
%\tikzset{linea/.style={-triangle 90}, }
%\tikzset{main node2/.style={rectangle, draw,thick,
%    text width=7em, text centered, rounded corners, minimum height=3em}, }
\tikzset{main node2/.style={main node,
    text width=7em, minimum height=3em}, }
\tikzset{main verb/.style={minimum size=1cm}, }
\tikzset{linea/.style={-triangle 90,thick,draw}}
\tikzset{primo/.style={circle,draw,inner sep=0pt,minimum size=1pt,thick}, }
\tikzset{
    start-end/.style={
        draw,
        rectangle,
        rounded corners,main base,
    } ,
    input/.style={ % requires library shapes.geometric
        draw,
        trapezium,
        trapezium left angle=60,
        trapezium right angle=120,main base
    },
    operation/.style={
        draw,thick,
        rectangle,main base,
    },
    loop/.style={ % requires library shapes.misc
        draw,
        chamfered rectangle,
        chamfered rectangle xsep=2cm
    },
    decision/.style={ % requires library shapes.geometric
        draw,
        diamond,
        aspect=#1,main base
    },
    decision/.default=1,
    print/.style={ % requires library shapes.symbols
        draw,
        tape,
        tape bend top=none
    },
    connection/.style={
        draw,
        circle,
        radius=5pt,
    },
    process rectangle outer width/.initial=0.15cm,
    predefined process/.style={
        rectangle,
        draw,
        append after command={
        \pgfextra{
          \draw
          ($(\tikzlastnode.north west)-(0,0.5\pgflinewidth)$)--
          ($(\tikzlastnode.north west)-(\pgfkeysvalueof{/tikz/process rectangle outer width},0.5\pgflinewidth)$)--
          ($(\tikzlastnode.south west)+(-\pgfkeysvalueof{/tikz/process rectangle outer width},+0.5\pgflinewidth)$)--
          ($(\tikzlastnode.south west)+(0,0.5\pgflinewidth)$);
          \draw
          ($(\tikzlastnode.north east)-(0,0.5\pgflinewidth)$)--
          ($(\tikzlastnode.north east)+(\pgfkeysvalueof{/tikz/process rectangle outer width},-0.5\pgflinewidth)$)--
          ($(\tikzlastnode.south east)+(\pgfkeysvalueof{/tikz/process rectangle outer width},0.5\pgflinewidth)$)--
          ($(\tikzlastnode.south east)+(0,0.5\pgflinewidth)$);
        }  
        },
        text width=#1,
        align=center
    },
    predefined process/.default=1.75cm,
    man op/.style={ % requires library shapes.geometric
        draw,
        trapezium,
        shape border rotate=180,
        text width=2cm,
        align=center,
    },
    extract/.style={
        draw,
        isosceles triangle,
        isosceles triangle apex angle=60,
        shape border rotate=90
    },
    merge/.style={
        draw,
        isosceles triangle,
        isosceles triangle apex angle=60,
        shape border rotate=-90
    },
}
\begin{document}
  \begin{tikzpicture}
    \node[main node2] (1) {Frazione};
    \node[main verb] (4) [right=2cm of 1]  {ha};
    \node[main verb] (3) [above=2cm of 4]  {ha};
     \node[main verb] (2) [above=2cm of 3]  {ha};
    \node[main verb] (5) [below= of 4]  {ha};
    \node[main verb] (6) [below= of 5]  {ha};
    \node[main node2] (7) [right= of 2] {Al denominatore c'è un numero formato non solo da potenze del 2 e del 5};
    \node[main node2] (8) [right= of 3] {Al denominatore c'è un numero che non è formato da potenze del 2 e del 5};
    \node[main node2] (9) [right= of 4] {Al denominatore c'è un numero che è formato da potenze del 2 e del 5};
    \node[main node2] (10) [right= of 5] {Denominatore decimale};
    \node[main node2] (11) [right= of 6] {Frazione impropria};
    \node[main verb] (12) [right= of 7]  {\'{e}};
   \node[main verb] (13) [right= of 8]  {\'{e}};
   %\node[main verb] (14) [right= of 9]  {\'{e}};
       \path let \p1=(9), \p2=(10) in node[main verb] (14) at (\x1+80,\y2/2+\y1/2){\'{e}};
   \node[main verb] (15) [right= of 11]  {\'{e}};
    \node[main node2] (16) [right= of 12] {Numero decimale periodico composto};
    \node[main node2] (17) [right= of 13] {Numero decimale periodico semplice};
    \node[main node2] (18) [right= of 14] {Numero decimale finito};
    \node[main node2] (19) [right= of 15] {Numero intero};
 
\foreach \x /\y in{1/2,1/3,1/4,1/5,1/6,2/7,3/8,4/9,5/10,6/11,7/12,8/13,9/14,10/14,11/15,12/16,13/17,14/18,15/19}
  \path[linea] (\x) edge node {} (\y);
%  
\end{tikzpicture}
\end{document}