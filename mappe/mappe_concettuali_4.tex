\documentclass{standalone}
\input{../Mod_base/grafica}
\tikzset{main node/.style={rectangle, draw,thick,
    text width=7em, text centered, rounded corners, minimum height=4em}, }
\tikzset{main verb/.style={minimum size=1cm}, }
\tikzset{linea/.style={-triangle 90,thick}, }
\begin{document}
  \begin{tikzpicture}
    \node[main node] (1) {Moltiplicazione};
    \node[main verb] (2) [right =of 1]  {\'{e}};
    \node[main node] (3) [right =of 2] {Operazione};
\node[main verb] (4) [above right= 1.5cm and 1cm  of 3]  {ha};
\node[main verb] (5) [below right= .0cm and 1cm of 3]  {\'{e}};
\node[main verb] (6) [above  =1.5cm and .1cm of 4]  {\'{e} formata};
\node[main verb] (7) [below  =4cm and 1cm of 5]  {ha};

\node[main node] (10) [above right= .8cm and 1cm  of 5]  {Distributiva};
\node[main node] (11) [below right= .8cm and 1cm of 5]  {Associativa};
\node[main node] (12) [right  =.8cm and 1cm of 5]  {Commutativa};
\node[main node] (9) [right  =.4cm and 1cm of 4]  {Un risultato};
\node[main node] (8) [right  =.4cm and .7cm of 6]  {Da fattori};
\node[main node] (13) [above right= .1cm and 1cm  of 7]  {Un elemento assorbente};
\node[main node] (14) [below right= .1cm and 1cm of 7]  {Un elemento neutro};
\node[main verb] (15) [right =of 9]  {\'{e}};
\node[main verb] (16) [right =of 13] {\'{e}};
\node[main verb] (17) [right =of 14] {\'{e}};
\node[main node] (18) [right =of 15] {Il prodotto};
\node[main node] (19) [right =of 16] {Lo zero};
\node[main node] (20) [right =of 17] {Uno};
\foreach \x /\y in{1/2,2/3,3/6,3/4,3/5,3/7,6/8,4/9,5/10,5/11,5/12,7/13,7/14,15/18,16/19,17/20,13/16,14/17,9/15}
  \path[linea] (\x) edge node {} (\y);
%  
\end{tikzpicture}
\end{document}