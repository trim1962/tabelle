\chapter{Equazioni}
\label{sec:equazioni}
\section{Definizioni}
Un'equazione\index{Equazione} è l'uguaglianza fra due espressioni. Dato che dipende dai valori che vengono assegnati alle lettere un'equazione è un'uguaglianza condizionata. I valori, che sostituiti alle lettere rendono vere l'uguaglianza,  sono chiamate soluzioni\index{Equazione!soluzione} . 
\begin{esempio}
\begin{itemize}
\item $2+3=3+5$ non è un'equazione. Mancano le lettere.
\item $2+x=3x+5y$ è un'equazione. 
\item $z+3x=0$ è un'equazione.
\end{itemize}
\end{esempio}
\begin{esempio}
Data l'equazione $2x-5=x+2$ $x=7$ è soluzione infatti $2\cdot 7-5=7+2$ $9=9$ l'uguaglianza è verificata. Mentre per $x=3$ $2\cdot 3-5=7+2$ $1\neq5$ quindi non è soluzione.
\end{esempio}
Le lettere sono chiamate variabili\index{Equazione!variabile} o costanti\index{Equazione!costante}. Una variabile o incognita è una quantità non nota a priori che può assumere qualunque valore. Una costante è una quantità non nota ma fissa. Di solito si usano $x,y,z$ per indicare le incognite $a,b,c,\dots$ per indicare le costanti.  

Un'equazione in cui compare una sola lettera è detta in una incognita, con due diverse in due incognite eccetera. Il segno di uguaglianza divide l'equazione in due parti, la parte a sinistra chiamata primo membro, la parte a destra secondo membro.
\section{Principi di equivalenza}
Due o più equazioni sono equivalenti\index{Equazione!equivalente} se hanno le stesse soluzioni.
\begin{esempio}
Le equazioni 
\begin{align*}
5x+4&=4x+3\\
3x+2&=2x+3
\end{align*}
hanno la stessa soluzione
\begin{align*}
5x+4&=4x+3\\
5\cdot(-1)+4&=4\cdot(-1)+3\\
-1&=-1\\
3x+2&=2x+3\\
3\cdot(-1)+2&=2\cdot(-1)+3\\
-1&=-1
\end{align*}
Quindi le due equazioni sono equivalenti.
\end{esempio}
\subsection{Primo principio di equivalenza}
\label{sec:PrimoprincipioEquivalenza}
Se aggiungiamo o togliamo la stessa quantità\footnote{Quantità definita} al primo e al secondo membro di una equazione,  otteniamo un'equazione  equivalente\index{Equazione!equivalente} a quella di partenza.
\begin{esempio}
\begin{align*}
8x+14&=6x+10
\intertext{aggiungendo $+10$ ad entrambi i membri}
8x+14+10&=6x+10+10\\
8x+24&=6x+20
\intertext{le due equazioni sono equivalenti infatti}
8\cdot(-2)+14&=6\cdot(-2)+10\\
8&=8\\
8\cdot(-2)+14&=6\cdot(-2)+10\\
-2&=-2
\intertext{quindi $-2$ è soluzione per entrambe}
\end{align*}
\end{esempio}
\subsection{Conseguenze primo principio}
Se un termine passa  dal primo al secondo membro di una equazione o viceversa cambia di segno.
\begin{esempio}
\begin{NodesList}[dy=5pt,margin=3cm]
 \[ % formula no "inline"
 \begin{aligned}
 x+5 &= 8 \AddNode\\
 x +5-5 &= 8-5 \AddNode\\
 x + 0 &= 8-5 \AddNode
 \end{aligned}
 \]
 \LinkNodes{\begin{minipage}{2cm}
aggiungo $-5$ ad entrambi i membri
 \end{minipage}
 }
 \LinkNodes{ $5-5=0$ }
 \end{NodesList}
 in pratica
 \begin{NodesList}[dy=5pt,margin=3cm]
  \[ % formula no "inline"
  \begin{aligned}
  x+5 &= 8 \AddNode\\
  x  &= 8-5 \AddNode
  \end{aligned}
  \]
  \LinkNodes{\begin{minipage}{2cm}
 sposto e cambio di segno
  \end{minipage}
  }
  \end{NodesList}
\end{esempio}
Se la stessa quantità è presente nel primo o secondo membro dell'equazione allora può essere eliminata.
\begin{esempio}
\begin{NodesList}[dy=5pt,margin=3cm]
 \[ % formula no "inline"
 \begin{aligned}
 x+5 &= 8+5 \AddNode\\
 x +5-5 &= 8+5-5 \AddNode\\
 x + 0 &= 8+0 \AddNode
 \end{aligned}
 \]
 \LinkNodes{\begin{minipage}{2cm}
aggiungo $-5$ ad entrambi i membri
 \end{minipage}
 }
 \LinkNodes{ $5-5=0$ }
 \end{NodesList}
 in pratica
 \begin{NodesList}[dy=5pt,margin=3cm]
  \[ % formula no "inline"
  \begin{aligned}
  x+5 &= 8+5 \AddNode\\
  x  &= 8 \AddNode
  \end{aligned}
  \]
  \LinkNodes{\begin{minipage}{2cm}
semplifico
  \end{minipage}
  }
  \end{NodesList}
\end{esempio}
\subsection{Secondo principio di equivalenza}
\label{sec:SecondoprincipioEquivalenza}
Se moltiplichiamo o dividiamo per  la stessa quantità diversa da zero il primo e il secondo membro di una equazione,  otteniamo un'equazione  equivalente\index{Equazione!equivalente} a quella di partenza.
\begin{esempio}
\begin{align*}
2x+2&=x+5
\intertext{moltiplico per  $+5$  entrambi i membri}
5\cdot(2x+2)&=5\cdot(x+5)\\
10x+10&=5x+25
\intertext{le due sono equivalenti infatti}
2\cdot(3)+2&=3+5\\
8&=8\\
10\cdot(3)+10&=5\cdot(3)+25\\
40&=40
\intertext{quindi $-2$ è soluzione per entrambe}
\end{align*}
\end{esempio}
%\begin{table}[H]
%\centering
%\begin{tabular}{LCR}
%\toprule
%+a&=&\ldots\\
%\ldots&=&-a\\
%\bottomrule
%\end{tabular}
%\caption{Regola del trasporto}
%\label{tab:regtrasporto}
%\end{table}
%\begin{table}[H]
%\centering
%\begin{tabular}{LCR}
%\toprule
%\dfrac{\cdots\cdots}{a}&=&\dfrac{\cdots\cdots}{a}\\
%&\\
%a\cdot\dfrac{\cdots\cdots}{a}&=&a\cdot\dfrac{\cdots\cdots}{a}\\
%&\\
%\cdots\cdots&=&\cdots\cdots\\
%\bottomrule
%\end{tabular}
%\caption{Semplificazione denominatore}
%\label{tab:Semplificazionedenominatore}
%\end{table}
%\begin{table}[H]
%
%\centering
%\begin{tabular}{CCCCL}
%\toprule
%\multicolumn{5}{c}{ax=b}\\
%\hline
%%&\\
%\multicolumn{2}{c}{coefficienti}&&soluzione&tipo soluzione\\
%\midrule
%a\neq0&b\neq0&ax=b&x=\dfrac{b}{a}&determinata\\
%%&\\
%a\neq0&b=0&ax=0&x=0&determinata\\
%%&\\
%a=0&b=0&0x=0&&indeterminata\\
%%&\\
%a=0&b\neq0&0x=b&&impossibile\\
%\bottomrule	
%\end{tabular}
%\caption{Soluzioni equazioni primo grado intere}
%\label{tab:equazioniprimogrado}
%\end{table}
%\begin{table}%
%
%\centering
%\begin{tabular}{LR}
%\toprule
%Tipo&Nome\\
%\midrule
%ax^2+c=0&Pura\\
%\hline
%\multicolumn{2}{c}{Risoluzione}\\
%\multicolumn{2}{C}{ax^2=-c}\\
%\multicolumn{2}{C}{x^2=-\dfrac{c}{a}}\\
%\multirow{3}*{Se $-\dfrac{c}{a}>0$ esistono soluzioni reali} &x_1=-\sqrt{-\dfrac{c}{a}}\\
%&\\
%&x_2=+\sqrt{-\dfrac{c}{a}}\\
%&\\
%Se -\dfrac{c}{a}<0\text{ non esistono soluzioni reali}&\\
%&\\
%\bottomrule	
%%\end{tabular}
%%\caption{Equazione secondo grado pura}
%%\label{tab:equazione2GradoPura}
%%\end{table}
%%\begin{table}%
%%
%%\centering
%%\begin{tabular}{LR}
%\toprule
%Tipo&Nome\\
%\midrule
%ax^2+bx=0&Spuria\\
%\hline
%\multicolumn{2}{c}{Risoluzione}\\
%\multicolumn{2}{C}{ax^2+bx=0}\\
%\multicolumn{2}{C}{x(ax+b)=0}\\
%\multicolumn{2}{C}{x_1=0}\\
%\multicolumn{2}{C}{ax+b=0}\\
%\multicolumn{2}{C}{x_2=-\dfrac{b}{a}}\\
%\bottomrule	
%%\end{tabular}
%%\caption{Equazione secondo grado spuria}
%%\label{tab:equazione2GradoSpuria}
%%\end{table}
%%\begin{table}%
%%
%%\centering
%%\begin{tabular}{LR}
%\toprule
%Tipo&Nome\\
%\midrule
%ax^2=0&Monomia\\
%\hline
%\multicolumn{2}{c}{Risoluzione}\\
%\multicolumn{2}{C}{ax^2=0}\\
%\multicolumn{2}{C}{x_1=0}\\
%\multicolumn{2}{C}{x_2=0}\\
%\bottomrule	
%%\end{tabular}
%%\caption{Equazione secondo grado monomia}
%%\label{tab:equazione2GradoMonomia}
%%\end{table}
%%\begin{table}%
%%
%%\centering
%%\begin{tabular}{LR}
%\toprule
%Tipo&Nome\\%
%\midrule
%ax^2+bx+c=0&Completa\\%
%\hline
%\multicolumn{2}{c}{Risoluzione}\\%
%\multirow{3}*{$b^2-4ac>0$}&x_1=\dfrac{-b+\sqrt{b^2-4ac}}{2a}\\%
%&\\
%&x_2=\dfrac{-b-\sqrt{b^2-4ac}}{2a}\\%
%\hline
%\multirow{3}*{$b^2-4ac=0$}&x_1=-\dfrac{b}{2a}\\%
%&\\
%&x_2=-\dfrac{b}{2a}\\%
%\hline
%\multirow{3}*{$b^2-4ac<0$}&\\
%&\text{nessuna soluzione reale}\\%
%&\\
%\bottomrule	
%\end{tabular}
%\caption{Equazioni secondo grado}
%\label{tab:equazione2Gradoelenco}
%\end{table}
