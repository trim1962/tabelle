\documentclass[preview=true]{standalone}
\input{../Mod_base/grafica}
\input{../Mod_base/base}

%\usetikzlibrary{...}
\begin{document}
	

\begin{tikzpicture}[>=triangle 45]
% draw the coordinates
\pgfmathsetmacro{\raggio}{4};
\pgfmathsetmacro{\pangolo}{270};
\pgfmathsetmacro{\sangolo}{{-90}};
\pgfmathsetmacro{\mraggio}{\raggio/3};
\pgfmathsetmacro{\sraggio}{1.9*\raggio};
% draw the unit circle
\draw[->] (0,-\raggio-\mraggio) -- (0,\raggio+\mraggio) node[above] {$y$};
\draw[->] (-\raggio-\mraggio,0) -- (\raggio+\mraggio,0) node[right] {$x$};
\draw (-\raggio-\mraggio,-\raggio) -- (\raggio+\mraggio,-\raggio) ;
\draw[thick] (0,0) circle(\raggio);
\coordinate [label= below left:$O$] (OO)at(0,0);
\coordinate (P)  at ({\raggio*cos(\pangolo},{\raggio*sin(\pangolo});
\node at (P) [label=above right:$m$]{};
\fill [color=black] (OO) circle (1.5pt);
\fill [color=black] (P) circle (1.5pt);
\draw [->] (0:\sraggio/4) arc (0:\pangolo:\sraggio/4) ;
\draw (\pangolo/2:\sraggio/3.5) node[ left]  {$\ang{270}$};
\draw [->] (0:\sraggio/4) arc (0:\sangolo:\sraggio/4) ;
\draw (\sangolo/2:\sraggio/4) node[above left]  {$\ang{-90}$};
\end{tikzpicture}  
	
\end{document}