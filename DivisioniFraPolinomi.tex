\chapter{Divisioni fra polinomi}
\label{cha:Divisionipolinomi}
\minitoc
\mtcskip                                % put some skip here
\minilof                                % a minilof
\mtcskip                                % put some skip here
\minilot
\section{Divisioni fra monomi}
Un monomio è divisibile per un altro monomio se il grado del dividendo per ciascuna lettera è minore o uguale al grado della stessa lettera del divisore.
\begin{esempio}
Le seguenti divisioni sono possibili
\begin{align*}
3x^3y^2:x^y=&3x^0y^1=3y\\
4x^5a^2b:2x^2a=&2x^3ab
\end{align*}
La seguente divisione è impossibile
\begin{align*}
x^4y^3:y^5=&x^4y^{-2}
\end{align*}
\end{esempio}
\section{Divisioni fra Polinomi e monomi}
La divisione di un polinomio per un monomio è possibile se è possibile la divisione fra ogni termine del polinomio con il monomio.
\section{Divisione fra polinomi}
\begin{esempio}
Supponiamo di voler fare la seguente divisione $(x^3-x^4+1):(x^2+1)$
\begin{NodesList}
\begin{align*}
(x^3-x^4+1):(x^2+1)&\AddNode\\
&\\
(-x^4+x^3+1):(x^2+1)&\AddNode\\
&\\
\begin{minipage}[t]{0.5\textwidth}
\begin{tabular}{lllll|l}
$-x^4$& $+x^3$ & \phantom{$-x^2$} &\phantom{$-x$}  &$+1$&$ x^2+1$\\ 
\cline{6-6}&&&&&
%  \vrule height 2.5ex width 0pt $-x^4$& &$-x^2$  &  &&$-x^2+x+1$\\ 
%\cline{1-5}
%  \vrule height 2.5ex width 0pt &$+x^3$ & $+x^2$ &  &$+1$&  \\ 
% &$+x^3$& &$+x$&&  \\ 
%\cline{2-5}
%   \vrule height 2.5ex width 0pt&  &$+x^2$&$-x$&$+1$&  \\ 
%   \vrule height 2.5ex width 0pt&  &$+x^2$&&$+1$&  \\ 
%\cline{3-5}
% &  &  &$-x$&&  \\ 
\\
\end{tabular}
\end{minipage}&\AddNode\\
\begin{minipage}[t]{0.5\textwidth}
\begin{tabular}{lllll|l}
$-x^4$& $+x^3$ &\phantom{$-x^2$}  &\phantom{$-x$}  &$+1$&$ x^2+1$\\ 
\cline{6-6}&&&&&$-x^2$
%  \vrule height 2.5ex width 0pt $-x^4$& &$-x^2$  &  &&$-x^2+x+1$\\ 
%\cline{1-5}
%  \vrule height 2.5ex width 0pt &$+x^3$ & $+x^2$ &  &$+1$&  \\ 
% &$+x^3$& &$+x$&&  \\ 
%\cline{2-5}
%   \vrule height 2.5ex width 0pt&  &$+x^2$&$-x$&$+1$&  \\ 
%   \vrule height 2.5ex width 0pt&  &$+x^2$&&$+1$&  \\ 
%\cline{3-5}
% &  &  &$-x$&&  \\ 
\\
\end{tabular}
\end{minipage}&\AddNode\\
\begin{minipage}[t]{0.5\textwidth}
\begin{tabular}{lllll|l}
$-x^4$& $+x^3$ & \phantom{$-x^2$} & \phantom{$-x$} &$+1$&$ x^2+1$\\ 
\cline{6-6}
  \vrule height 2.5ex width 0pt $-x^4$& &$-x^2$  &  &&$-x^2$\\ 
\cline{1-5}
  \vrule height 2.5ex width 0pt &$+x^3$ & $+x^2$ &  &$+1$&  \\ 
% &$+x^3$& &$+x$&&  \\ 
%\cline{2-5}
%   \vrule height 2.5ex width 0pt&  &$+x^2$&$-x$&$+1$&  \\ 
%   \vrule height 2.5ex width 0pt&  &$+x^2$&&$+1$&  \\ 
%\cline{3-5}
% &  &  &$-x$&&  \\ 
\\
\end{tabular}
\end{minipage}&\AddNode\\
\begin{minipage}[t]{0.5\textwidth}
\begin{tabular}{lllll|l}
$-x^4$& $+x^3$ & \phantom{$-x^2$} & \phantom{$-x$} &$+1$&$ x^2+1$\\ 
\cline{6-6}
  \vrule height 2.5ex width 0pt $-x^4$& &$-x^2$  &  &&$-x^2+x$\\ 
\cline{1-5}
  \vrule height 2.5ex width 0pt &$+x^3$ & $+x^2$ &  &$+1$&  \\ 
% &$+x^3$& &$+x$&&  \\ 
%\cline{2-5}
%   \vrule height 2.5ex width 0pt&  &$+x^2$&$-x$&$+1$&  \\ 
%   \vrule height 2.5ex width 0pt&  &$+x^2$&&$+1$&  \\ 
%\cline{3-5}
% &  &  &$-x$&&  \\ 
\\
\end{tabular}
\end{minipage}&\AddNode\\
\begin{minipage}[t]{0.5\textwidth}
\begin{tabular}{lllll|l}
$-x^4$& $+x^3$ & \phantom{$-x^2$} & \phantom{$-x$} &$+1$&$ x^2+1$\\ 
\cline{6-6}
  \vrule height 2.5ex width 0pt $-x^4$& &$-x^2$  &  &&$-x^2+x$\\ 
\cline{1-5}
  \vrule height 2.5ex width 0pt &$+x^3$ & $+x^2$ &  &$+1$&  \\ 
 &$+x^3$& &$+x$&&  \\ 
\cline{2-5}
   \vrule height 2.5ex width 0pt&  &$+x^2$&$-x$&$+1$&  \\ 
%   \vrule height 2.5ex width 0pt&  &$+x^2$&&$+1$&  \\ 
%\cline{3-5}
% &  &  &$-x$&&  \\ 
\\
\end{tabular}
\end{minipage}&\AddNode\\
\begin{minipage}[t]{0.5\textwidth}
\begin{tabular}{lllll|l}
$-x^4$& $+x^3$ & \phantom{$-x^2$} & \phantom{$-x$} &$+1$&$ x^2+1$\\ 
\cline{6-6}
  \vrule height 2.5ex width 0pt $-x^4$& &$-x^2$  &  &&$-x^2+x+1$\\ 
\cline{1-5}
  \vrule height 2.5ex width 0pt &$+x^3$ & $+x^2$ &  &$+1$&  \\ 
 &$+x^3$& &$+x$&&  \\ 
\cline{2-5}
   \vrule height 2.5ex width 0pt&  &$+x^2$&$-x$&$+1$&  \\ 
%   \vrule height 2.5ex width 0pt&  &$+x^2$&&$+1$&  \\ 
%\cline{3-5}
% &  &  &$-x$&&  \\ 
\\
\end{tabular}
\end{minipage}&\AddNode\\
\begin{minipage}[t]{0.5\textwidth}
\begin{tabular}{lllll|l}
$-x^4$& $+x^3$ & \phantom{$-x^2$} & \phantom{$-x$} &$+1$&$ x^2+1$\\ 
\cline{6-6}
  \vrule height 2.5ex width 0pt $-x^4$& &$-x^2$  &  &&$-x^2+x+1$\\ 
\cline{1-5}
  \vrule height 2.5ex width 0pt &$+x^3$ & $+x^2$ &  &$+1$&  \\ 
 &$+x^3$& &$+x$&&  \\ 
\cline{2-5}
   \vrule height 2.5ex width 0pt&  &$+x^2$&$-x$&$+1$&  \\ 
   \vrule height 2.5ex width 0pt&  &$+x^2$&&$+1$&  \\ 
\cline{3-5}
 &  &  &$-x$&&  \\ 
\\
\end{tabular}
\end{minipage}&\AddNode\\
\end{align*}
\LinkNodes{Ordino i polinomi\\ }
\LinkNodes{\begin{minipage}{3.5cm}

Scrivo la divisione lasciando spazi vuoti dove necessario\\
\end{minipage}}%
  \LinkNodes{\begin{minipage}{3.5cm}
  
 \[\dfrac{-x^4}{x^2}=-x^2\]
  \end{minipage}
}%
 \LinkNodes{Calcolo il primo resto}%
  \LinkNodes{$\dfrac{x^3}{x^2}=x$}%
\LinkNodes{\begin{minipage}{3.5cm}
Calcolo il secondo resto
\end{minipage}}%
   \LinkNodes{$\dfrac{x^2}{x^2}=1$}%
    \LinkNodes{Calcolo l'ultimo resto}%
   \end{NodesList}
\end{esempio}
\begin{esempio}
Supponiamo di voler dividere \[(x^4+2x+1):(x^2+1)\]
\begin{figure}
\rotatebox{90}{
\begin{minipage}[b]{.35\linewidth}
\centering\includestandalone[width=5.5cm]{polinomi/divpolinomi7}
\subcaption{Sette}\label{fig:divpol2g}
\end{minipage}%
}
\rotatebox{90}{
\begin{minipage}[b]{.35\linewidth}
\centering\includestandalone[width=5.5cm]{polinomi/divpolinomi6}
\subcaption{Sei}\label{fig:divpol2f}
\end{minipage}%
}
\rotatebox{90}{
\begin{minipage}[b]{.35\linewidth}
\centering\includestandalone[width=5.5cm]{polinomi/divpolinomi5}
\subcaption{Cinque}\label{fig:divpol2e}
\end{minipage}%
}
\rotatebox{90}{
\begin{minipage}[b]{.35\linewidth}
\centering\includestandalone[width=5.5cm]{polinomi/divpolinomi4}
\subcaption{Quattro}\label{fig:divpol2d}
\end{minipage}%
}
\rotatebox{90}{
\begin{minipage}[b]{.35\linewidth}
\centering\includestandalone[width=5.5cm]{polinomi/divpolinomi3}
\subcaption{Tre}\label{fig:divpol2c}
\end{minipage}
}
\rotatebox{90}{
\begin{minipage}[b]{.35\linewidth}
\centering\includestandalone[width=5.5cm]{polinomi/divpolinomi2}
\subcaption{Due}\label{fig:divpol2b}
\end{minipage}
}
\rotatebox{90}{
\begin{minipage}[b]{.35\linewidth}
\centering\includestandalone[width=5.5cm]{polinomi/divpolinomi1}
\subcaption{Uno}\label{fig:divpol2a}
\end{minipage}%
}
\caption{Divisione fra polinomi}\label{fig:divpolinomi2}
\end{figure}
\end{esempio}
\section{Metodo di Ruffini}
\begin{esempio}
Supponiamo di voler dividere
\[(x^2+2x+1):(x+1)\]
\begin{figure}
	\begin{subfigure}[b]{0.55\linewidth}
		\centering\includestandalone[width=0.6\textwidth]{polinomi/ruffini1}
		\caption{Imposto il castello}\label{fig:Ruffiniesempio1a}
	\end{subfigure}%
	\captionsetup{skip=0pt}
	\begin{subfigure}[b]{0.55\linewidth}
		\centering\centering\includestandalone[width=0.6\textwidth]{polinomi/ruffini2}
		\caption{Sposto il coefficiente sotto la riga}\label{fig:Ruffiniesempio1b}
	\end{subfigure}
	\begin{subfigure}[b]{.55\linewidth}
		\centering\centering\includestandalone[width=0.6\linewidth]{polinomi/ruffini3}
		\caption{moltiplico e sposto}\label{fig:Ruffiniesempio1c}
	\end{subfigure}%
		\captionsetup{skip=0pt}
	\begin{subfigure}[b]{.55\linewidth}
		\centering\centering\includestandalone[width=0.6\textwidth]{polinomi/ruffini4}
		\caption{Sommo sulla colonna}\label{fig:Ruffiniesempio1d}
	\end{subfigure}
	\begin{subfigure}[b]{.55\linewidth}
			\centering\centering\includestandalone[width=0.6\textwidth]{polinomi/ruffini5}
			\caption{Moltiplico e sposto la risposta}\label{fig:Ruffiniesempio1e}
		\end{subfigure}%
		\captionsetup{skip=0pt}
		\begin{subfigure}[b]{.55\linewidth}
			\centering\centering\includestandalone[width=0.6\textwidth]{polinomi/ruffini6}
			\caption{Sommo sulla colonna fine}\label{fig:Ruffiniesempio1f}
		\end{subfigure}
	\captionof{figure}{Metodo di Ruffini}
\label{fig:Ruffiniesempio1}
\end{figure}


la risposta è \[x+1\] con resto zero.
\end{esempio}
