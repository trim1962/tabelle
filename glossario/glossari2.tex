%a

%C
\newglossaryentry{costanteg}{name={Costante},
description={una costante è un carattere che rappresenta una quantità numerica non nota ma fissato}}
%D

%E
\newglossaryentry{equazioneg}{name={Equazione},
description={uguaglianza condizionata fra due espressioni algebriche}}
\newglossaryentry{equazioneDetg}{name={Equazione determinata},
description={è un'equazione con un numero infinito di soluzioni}}
\newglossaryentry{equazioneImpg}{name={Equazione impossibile},
description={è un'equazione che non ha soluzioni}}
\newglossaryentry{equazioneIdeng}{name={Equazione identità},
description={è un'equazione sempre verificata}}
\newglossaryentry{equazioneIdentg}{name={Equazione indeterminata},
description={è un'equazione con un numero infinito di soluzioni}}
\newglossaryentry{equazioneequig}{name={Equazioni equivalenti},
description={sono equazioni che hanno le stesse soluzioni}}
%F
\newglossaryentry{equazionenormaleg}{name={Forma normale},
description={una equazione è in forma normale se tutti i termini sono a primo membro ordinati}}
%I

%M

%N

%P
\newglossaryentry{primoprincipioequig}{name={Primo principio equivalenza},
description={sommando o sottraendo la stessa quantità al primo e al secondo membro di una equazione si ottiene una equazione equivalente a quella data}}
\newglossaryentry{primomembroequazioneg}{name={Primo membro},
description={in una equazione è il termine che si trova a sinistra del segno gi uguaglianza}}

%Q

%R
\newglossaryentry{risoluzioneequag}{name={Risolvere un'equazione},
description={trovare le soluzione dell'equazione}}
%S
\newglossaryentry{secondoprincipioequig}{name={Secondo principio equivalenza},
description={moltiplicando o dividendo per  la stessa quantità diversa da zero il primo e il secondo membro di una equazione si ottiene una equazione equivalente a quella data}}
\newglossaryentry{soluzionequazioneg}{name={Soluzione},
description={valore che sostituito alle incognite rende vera l'equazione}}
\newglossaryentry{secondomembroequazioneg}{name={Secondo membro},
description={in una equazione è il termine che si trova a destra del segno gi uguaglianza}}

%V
\newglossaryentry{variabileg}{name={Variabile},
description={una variabile è un carattere che rappresenta una quantità numerica non nota}}

	