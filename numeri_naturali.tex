\chapter{Numeri Naturali}
\label{cha:NumeriNaturali}
\minitoc
\mtcskip                                % put some skip here
\minilof                                % a minilof
\mtcskip                                % put some skip here
\minilot
I numeri naturali\index{Numeri!Naturali} sono un insieme numerico, indicato con il simbolo $\N$. Questo insieme, $\N=\Set{0,1,2,3,\dots,}$ è  costituito da un numero infinito di elementi. Questi numeri hanno, tranne lo zero, un precedente ed un successivo. Possiamo, quindi definire per i numeri naturali un ordine\index{Numeri!Naturali!ordinati}.  L'insieme dei numeri naturali\index{Numeri!Naturali!discerti} è discreto nel senso che fra un numero e il suo successivo non vi è nessun altro elemento dell'insieme. Una rappresentazione dell'inseme  è la retta orientata\nobs\vref{fig:NumeriNaturaliRetta} dove i numeri sono ordinati dal minore al maggiore secondo il verso della retta. 
\section{Operazioni}
Prima di parlare di operazioni in $\N$ spendiamo due parole sul concetto di operazione. In matematica, un'operazione è una relazione che lega, in generale, due elementi a un elemento detto risultato\nobs\vref{fig:Nat_operazione_binaria}. Operazioni di questo tipo vengono dette binarie. Esistono anche operazioni unarie come per esempio il cambio di segno, il quadrato di un numero eccetera in questo caso abbiamo un elemento in ingresso e uno in uscita.   Le operazioni si dividono in interne o esterne a seconda che il risultato appartenga o no all'insieme dei valori in ingresso. L'ordine con cui sono scritte è importante. La regola prevede che vengano eseguite andando da sinistra verso destra. Per variare l'ordine di esecuzione sono introdotte le parentesi che indicano cosa debba essere eseguita per prima.
\begin{figure} 
	\centering
\includestandalone[width=.9\textwidth]{numeri_naturali/N_rettaOrientata}
	\caption{Retta orientata}
	\label{fig:NumeriNaturaliRetta}\end{figure}
\begin{figure} 
	\centering
\includestandalone[width=.9\textwidth]{mappe/mappe_concettuali_1}
	\caption{Numeri Naturali}
	\label{fig:NumeriNaturali}\end{figure}
\begin{figure} 
	\centering
\includestandalone[width=.6\textwidth]{numeri_naturali/Operazione_binaria_}
	\caption{Operazione Binaria}
	\label{fig:Nat_operazione_binaria}\end{figure}
\begin{figure} 
	\centering
\includestandalone[width=.6\textwidth]{numeri_naturali/Operazione_unaria_}
	\caption{Operazione Unaria}
	\label{fig:Nat_operazione_unaria}\end{figure}
\begin{figure}
	\begin{subfigure}[b]{.5\linewidth}
		\centering
\includestandalone[width=.6\textwidth]{numeri_naturali/Operazione_Addizione_}
	\caption{Operazione Addizione}
	\label{fig:Nat_operazione_Addizione}
	\end{subfigure}%
	\begin{subfigure}[b]{.5\linewidth}
	\centering
\includestandalone[width=.6\textwidth]{numeri_naturali/Operazione_Sottrazione_}
	\caption{Operazione Sottrazione}
	\label{fig:Nat_operazione_Sottrazione}
	\end{subfigure}
	\begin{subfigure}[b]{.5\linewidth}
	\centering
\includestandalone[width=.6\textwidth]{numeri_naturali/Operazione_Moltiplicazione_}
	\caption{Operazione Moltiplicazione}
	\label{fig:Nat_operazione_Moltiplicazione}
	\end{subfigure}
	\begin{subfigure}[b]{.5\linewidth}
	\centering
\includestandalone[width=.6\textwidth]{numeri_naturali/Operazione_Divisione_}
	\caption{Operazione Divisione}
	\label{fig:Nat_operazione_Divisione}
	\end{subfigure}
		\begin{subfigure}[b]{.5\linewidth}
		\centering
	\includestandalone[width=.6\textwidth]{numeri_naturali/Operazione_Potenza_}
		\caption{Operazione Potenza}
		\label{fig:Nat_operazione_Potenza}
		\end{subfigure}
	\captionof{figure}{Operazioni in $\N$}
	\label{fig:OPerzioni in N}
\end{figure}
\subsection{Addizione}
\label{sec:NumerinatADD}
L'addizione\index{Operazione!addizione} è una operazione binaria\nobs\vref{fig:Nat_operazione_Addizione}  interna in $\N$. I termini in ingresso si dicono addendi\index{Operazione!addizione!addendo}, il risultato somma\index{Operazione!addizione!somma}. L'operazione di  addizione è commutativa\index{Operazione!addizione!commutativa} cioè cambiando l'ordine degli addendi il risultato non cambia \[2+3=3+2=5\]. L'operazione ha un elemento neutro\index{Operazione!addizione!elemento neutro} lo zero. L'addizione dell'elemento neutro e di un addendo ha per somma l'addendo  \[4+0=0+4=4\]. L'addizione è associativa\index{Operazione!addizione!associativa}. Nell'addizione tre numeri, non conta l'ordine con cui vengano fatte le addizioni, il risultato non cambia\nobs\vref{fig:ProprietaAddizione}\[2+3+4=(2+3)+4=2+(3+4)\]
\begin{figure} %
	\centering
\includestandalone[width=.9\textwidth]{mappe/mappe_concettuali_2}
	\caption{Proprietà addizione}
	\label{fig:ProprietaAddizione}\end{figure}
\subsection{Sottrazione}
La sottrazione\index{Operazione!sottrazione} è una operazione binaria \nobs\vref{fig:Nat_operazione_Sottrazione} non sempre interna in $\N$. I termini in ingresso si dicono minuendo\index{Operazione!sottrazione!minuendo} e sottraendo\index{Operazione!sottrazione!sottraendo}, il risultato differenza\index{Operazione!sottrazione!differenza}. Se il sottraendo è maggiore del minuendo l'operazione è esterna. Se il minuendo è uguale al sottraendo la differenza è zero. La sottrazione non è commutativa \[3-2\neq2-3\] e neppure associativa \[(4-3)-2\neq4-(3-2)\]
 L'operazione gode della proprietà invariantiva\index{Operazione!sottrazione!invariantiva}, per cui aggiungendo o sottraendo la stessa quantità al minuendo e al sottraendo la differenza non cambia\nobs\vref{fig:ProprietaSottrazione}.  
\label{sec:NumerinatDiff}
\begin{figure} %
	\centering
\includestandalone[width=.9\textwidth]{mappe/mappe_concettuali_3}
	\caption{Proprietà Sottrazione}
	\label{fig:ProprietaSottrazione}\end{figure}
\subsection{Moltiplicazione}
\label{sec:NumerinatMolt}
\begin{figure} %
	\centering
\includestandalone[width=.9\textwidth]{mappe/mappe_concettuali_4}
	\caption{Proprietà Moltipplicazione}
	\label{fig:ProprietaMoltiplicazione}\end{figure}
La moltiplicazione\index{Operazione!moltiplicazione} è una operazione binaria\nobs\vref{fig:Nat_operazione_Moltiplicazione}  interna in $\N$. I termini in ingresso si dicono fattori\index{Operazione!moltiplicazione!fattore}, il risultato prodotto\index{Operazione!moltiplicazione!prodotto}. L'operazione di moltiplicazione  è commutativa\index{Operazione!moltiplicazione!commutativa} quindi cambiando l'ordine degli fattori il risultato non cambia \[2\cdot3=3\cdot2=6\]. L'operazione ha un elemento neutro\index{Operazione!moltiplicazione!elemento neutro} uno. La moltiplicazione dell'elemento neutro e di un fattore ha per prodotto il fattore  \[4\cdot1=1\cdot4=4\]. La moltiplicazione è associativa\index{Operazione!moltiplicazione!associativa}. Nella moltiplicazione di tre numeri o più numeri il risultato finale non cambia se vengono sostituiti due fattori con il loro prodotto\nobs\vref{fig:ProprietaMoltiplicazione} \[2\cdot3\cdot4=(2\cdot3)\cdot4=2\cdot(3\cdot4)\]. La moltiplicazione è dissociativa\index{Operazione!moltiplicazione!dissociativa}. Nella moltiplicazione  il risultato finale non cambia se viene sostituito un fattore con altri fattori il cui prodotto è uguale al fattore sostituito.  \[6\cdot5=2\cdot3\cdot5\]. L'elemento assorbente\index{Operazione!moltipliczione!elemento assorbente} è lo zero. La moltiplicazione di un numero qualunque per zero ha come prodotto zero\nobs\vref{fig:ProprietaMoltiplicazione}.
\subsection{Divisione}
\label{sec:Numerinatdiv}
\begin{figure} %
	\centering
\includestandalone[width=.9\textwidth]{mappe/mappe_concettuali_5}
	\caption{Proprietà Divisione}\label{sec:NumerinatDiff}
	\label{fig:ProprietaDivisione}\end{figure}
La divisione\index{Operazione!divisione} è una operazione binaria \nobs\vref{fig:Nat_operazione_Divisione} non sempre interna in $\N$. I termini in ingresso si dicono dividendo\index{Operazione!divisione!dividendo} e divisore\index{Operazione!divisione!divisore}, il risultato quoziente\index{Operazione!divisione!quoziente}. Se il dividendo non è  multiplo  del divisore l'operazione è esterna. La divisione non è commutativa \[3\div2\neq2\div3\] e neppure associativa \[(4\div3)\div2\neq4\div(3\div2)\]. L'operazione gode della proprietà invariantiva\index{Operazione!divisione!invariantiva}, per cui moltiplicando  o dividendo la stessa quantità diverso da zero, al dividendo e al divisore il quoziente non cambia\nobs\vref{fig:ProprietaDivisione}. Casi particolari sono \[1\div0\], \[0\div a\]. Nel primo caso la divisione è impossibile.
\subsection{Potenza}
\label{sec:NumerinatPot}
\begin{figure} %
	\centering
\includestandalone[width=.9\textwidth]{mappe/mappe_concettuali_6}
	\caption{Proprietà Potenza}
	\label{fig:ProprietaPotenza}\end{figure}
\section{Numeri primi e composti}
\label{sec:Numeriprimiecomposti}
Una possibile classificazione dei numeri naturali è la seguente
\begin{figure} %
	\centering
\includestandalone[width=.9\textwidth]{mappe/mappe_concettuali_7}
	\caption{Claddificazione}
	\label{fig:ProprietaClassificazioneNumNat}\end{figure}
Quindi, per esempio, dato che $18:2=9$ con resto zero avremo:
\begin{enumerate}
	\item 18 è multiplo di 2 secondo 9
	\item 18 è divisibile per 2
\end{enumerate}
\subsection{Scomposizione in fattori primi}
Per scomporre un numero naturale come prodotto di numeri primi  si può usare  l'algoritmo\nobs\vref{fig:numnatscposizionefattori}
\begin{figure}
\centering
\begin{tikzpicture}[auto, -stealth, thick]
% Definizione dei nodi e delle loro scatole
\tikzstyle{line} = [draw,->]
\node[ellipse,draw ]  (primo) {Inizio};
\node[rectangle,draw, text width=6cm,align=flush center, below of=primo, node distance=5em]  (secondo) {dividi il numero dato per
	il più piccolo primo che lo divide};
\node[diamond,draw, below of=secondo,node distance=10em]  (terzo) {il quoziente è uno?};
\node[ellipse, draw,below of=terzo,node distance=10em ] (quarto) {Fine};

% collegamento dei nodi
\begin{scope}[every path/.style=line]
\path  (primo) --  (secondo);
\path (secondo) edge  (terzo);
\path (terzo)  -- node  {SI} (quarto);
\path (terzo.west)  -|  node [near start] {NO} +(-7em,0) |- (secondo);
\end{scope}
\end{tikzpicture}
	\caption{Scomposizione in fattori primi}
	\label{fig:numnatscposizionefattori}
\end{figure}
	
Esempio 

\primedecomp{252}

quindi $252=2^2\cdot 3^2\cdot 7$

Esempio
\primedecomp{125}
$125=5^3$
	
\section{Massimo Comun Divisore}
\label{sec:macdNaturali}

Dati due o più numeri, il $\mcd$ è il numero\index{mcd} più grande che li divide tutti. Vi sono casi, in cui il $\mcd$ vale uno, perché uno è l'unico numero che li divide entrambi. Esempio di ciò sono il numero 8 e il numero nove, tranne l'uno non vi sono numeri naturali che li dividono entrambi. In questo caso si dice
 che i due numeri sono primi fra di loro.
 	
\begin{figure}
	\centering
\begin{tikzpicture}[node distance=8em,auto, -stealth, thick ,align=flush center,scale=0.4]
% Definizione dei nodi e delle loro scatole
\tikzstyle{line} = [draw,->]
\node[ellipse, draw,node distance=2em] (primo) {Inizio};
\node[rectangle,draw,below of=primo,text width=2cm](secondo) {Scomponi il numero in fattori primi};
\node[diamond,draw,below of=secondo,text width=2cm ]  (terzo) {I numeri dati sono finiti};
\node[rectangle,draw,below of=terzo,text width=2.2cm] (quarto) {Allineo le scomposizioni ottenute};
\node[diamond,draw,below of=quarto,text width=2cm]  (cinque) {Vi sono fattori in comune?};
\node[rectangle,draw,right of=cinque]  (sei) {mcd=1};
\node[rectangle,draw,below of=cinque,text width=2cm] (sette) {Prendo i fattori comuni con il minore esponenete};
\node[rectangle,draw,below of =sette,text width=2cm]  (otto) {mcd=prodotto dei fattori trovati};
\node[ellipse ,draw,below of=otto]  (nove) {Fine};
\begin{scope}[every path/.style=line]
\path  (primo) --  (secondo);
\path (secondo) edge  (terzo);
\path (terzo)  -- node  {SI} (quarto);
\path (terzo.west)  -|  node [near start]  {NO}+(-5em,0)  |-  (secondo);
\path(quarto)--(cinque);
\path (cinque.east)-- node [near start]{NO} (sei);
\path(cinque)--node[near start] {SI}(sette);
\path (sette)--(otto);
\path(otto)--(nove);
\end{scope}
\end{tikzpicture}
	\caption{Massimo Comun Divisore}
	\label{fig:numnatmcd}
\end{figure}

Esempio: per calcolare $\mcd(120,80,45)$ scompongo in fattori  primi i tre numeri come spiegato a\nobs\vref{fig:numnatscposizionefattori} e seguo lo schema\nobs\vref{fig:numnatmcd}     	
    \begin{tabular}{ccc}
    			\primedecomp{120}&\primedecomp{80}&\primedecomp{45}\\
    	\end{tabular}
    
    Allineo le scomposizioni
    
    \begin{tabular}{c}
    		$120=2^3\cdot 3\cdot5$\\
    	    $80=2^4\cdot 5$\\
    	    $45=3^2\cdot 5$
    \end{tabular}	
    
    I fattori comuni sono $2$ e $3$
    quindi $\mcd=2^3\cdot 3\cdot 5$
    \begin{figure}
    	\centering
    	\begin{tikzpicture}[auto, -stealth, thick, scale=0.5]
    	% Definizione dei nodi e delle loro scatole
    	\tikzstyle{line} = [draw,->]
    	\node[ellipse,draw] (zero) {Inizio};
    	\node[rectangle, draw,below of=zero, node distance=5em ]  (primo) {a,b};
    	\node[diamond,,draw,below of=primo,  node distance=5em]  (secondo) {b=0};
    	\node[rectangle,draw,right of= secondo,  node distance=10em]  (terzo) {mcd=a};
    	\node[rectangle,draw,below of= secondo,  node distance=5em]  (quarto) {a/b};
    	\node[diamond,,draw,below of=quarto,  node distance=5em]  (quinto) {resto=0};
    	\node[rectangle,draw,right of= quinto,  node distance=10em]  (sesto) {mcd=b};
    	\node[rectangle,draw,below of= quinto,  node distance=5em]  (settimo) {a=b b=r};
    	\node[ellipse,draw,below of= terzo,  node distance=5em]  (ottavo) {stop};
    	\node[ellipse,draw,below of=sesto,  node distance=5em]  (nono) {stop};
    	% collegamento dei nodi
    	\begin{scope}[every path/.style=line]
    	\path  (zero) edge  (primo);
    	\path  (primo) edge  (secondo);
    	%\path (secondo) edge  (terzo);
    	\path (secondo)  -- node  {SI} (terzo);
    	\path (secondo)  -- node  {NO} (quarto);
    	\path (quarto) edge  (quinto);
    	%\path (quinto) edge  (sesto);
    	\path (quinto)  -- node  {SI} (sesto);
    	\path (quinto)  -- node  {NO} (settimo);
    	\path (settimo.west)  -|  node [near start] {NO} +(-6em,0) |- (quarto);
    	\path (terzo) edge  (ottavo);
    	\path (sesto) edge  (nono);
    	\end{scope}
    	\end{tikzpicture}
    	\caption{Algoritmo di Euclide}
    	\label{fig:algoritmoEuclide}
    \end{figure}
   
   Un modo veloce per calcolare il $\mcd$ è l'algoritmo di Euclide\index{mcd!Euclide}. La figura\nobs\vref{fig:algoritmoEuclide} mostra la versione con divisione. Supponiamo di voler calcolare il $\mcd$ di $a=\num{27}$ e di $b=\num{15}$. Seguiamo lo schema, dividiamo $a$ con $b$ otteniamo un quoziente di 1 e un resto di $12$. Il resto non è $0$ per cui $a=\num{15}$ e di $b=\num{12}$ e ripetiamo. Dividiamo $a$ con $b$ otteniamo un quoziente di 1 e un resto di $3$. Il resto non è $0$ per cui $a=\num{15}$ e di $b=\num{3}$ e ripetiamo.  Dividiamo $a$ con $b$ otteniamo un quoziente di $5$ e un resto di $0$. Il resto è $0$ per cui $\mcd=\num{3}$    
   
   Supponiamo di voler calcolare il $\mcd$ fra \num{40} e \num{12}. Organizzo i calcoli come nella tabella\nobs\vref*{tab:euclidemcd1}. Inizio dividendo \num{40} per \num{12}. Ottengo come quoziente  \num{3} e per resto  \num{4}. La seconda riga ha per $a$ il precedente valore di $b$ e per $b$ il valore di $r$. Divido  \num{12} per \num{4}. Ottengo come quoziente  \num{3} e per resto  \num{0}. Essendo il resto uguale a zero, $\mcd(40;12)=4$
   \begin{table}
   	\centering
   	\begin{tabular}{SSSS}
   		\toprule
   		$a$&$b$&$a/b$&$r$\\
   		\midrule
   		40&12&3&4\\
   		12&4&3&0\\
   		\bottomrule
   	\end{tabular}
   	\caption[]{$\mcd$ \num{40} e \num{12}}
   	\label{tab:euclidemcd1}
   	\end{table} 
   	
   	Un esempio un po più lungo è nella tabella\nobs\vref*{tab:euclidemcd2}. In questo caso calcoliamo il $\mcd$ fra \num{85} e \num{26}. Dopo qualche passaggio otteniamo che il resto è zero quando $b=1$, quindi $\mcd(\num{85};\num{26})=1$. I numeri sono primi fra di loro. 
   	\begin{table}
   		\centering
   		\begin{tabular}{SSSS}
   			\toprule
   			$a$&$b$&$a/b$&$r$\\
   			\midrule
   			85&26&3&7\\
   			26&7&3&5\\
   			7&5&1&2\\
   			5&2&2&1\\
   			2&1&2&0\\
   			\bottomrule
   		\end{tabular}
   		\caption[]{$\mcd$ \num{85} e \num{26}}
   		\label{tab:euclidemcd2}
   	\end{table} 
    \section{Minimo comune multiplo}
        	\label{sec:mcmnumerinaturali}
     Il minimo comune multiplo fra due più o numeri è il più piccolo multiplo  in comune fra i numeri dati. 
    Esempio: per calcolare $\mcm(120,80,45)$ scompongo in fattori  primi i tre numeri come spiegato a\nobs\vref{fig:numnatscposizionefattori} e seguo lo schema\nobs\vref{fig:numnatmcm}     	
    	
    \begin{tabular}{ccc}
    		\primedecomp{120}&\primedecomp{80}&\primedecomp{45}\\
    	\end{tabular}
    	
    	Allineo le scomposizioni
    	
    	\begin{tabular}{c}
    		$120=2^3\cdot 3\cdot5$\\
    		$80=2^4\cdot 5$\\
    		$45=3^2\cdot 5$
    	\end{tabular}	
    	
    	quindi $\mcm=2^4\cdot 3^2\cdot 5$
    	
    	\begin{figure}
    		\centering
    		\begin{tikzpicture}[auto, -stealth, thick, scale=0.4]
    		% Definizione dei nodi e delle loro scatole
    		\tikzstyle{line} = [draw, -latex']
    		\node[ellipse, minimum height=4em, draw, very thick,inner xsep=3em] at (0,0) (primo) {Inizio
    		};
    		\node[rectangle,minimum height=4em,draw, very thick, text width=5cm,align=flush center] at (0,-4) (secondo) {Scomponi il numero in fattori primi};
    
    		\node[diamond,aspect=2,draw, very thick,inner xsep=3em, text width=2cm,align=flush center] at (0,-10
    		) (terzo) {I numeri dati sono finiti};
    		\node[rectangle,minimum height=4em,draw, very thick, text width=5cm,align=flush center] at (0,-17
    		) (quarto) {Allineo le scomposizioni ottenute};
    		\node[rectangle,minimum height=4em,draw, very thick, text width=5cm,align=flush center] at (0,-22)
    		(cinque) {Prendo i fattori comuni e non comuni con il maggiore esponente};
    		\node[rectangle,minimum height=4em,draw, very thick, text width=5cm,align=flush center] at (0,-27)
    		(sei) {mcm=prodotto dei fattori trovati};
    		
    		\node[ellipse ,minimum height=3em,draw, very thick,inner xsep=3em] at (0,-31) (sette) {Fine};
    			% collegamento dei nodi
    		\begin{scope}[every path/.style=line]
    		\path  (primo) --  (secondo);
    		\path (secondo) edge  (terzo);
    		\path (terzo)  -- node  {SI} (quarto);
    		\path (terzo.west)  -|  node [near start] {NO}+(-5em,0) |- (secondo);
    	
    		\path(quarto)--(cinque);
    		\path(cinque)--(sei);
    		\path(sei)--(sette);
    		
    		\end{scope}
    		\end{tikzpicture}
    			\caption{$\mcm$}
    			\label{fig:numnatmcm}
    		\end{figure}
  